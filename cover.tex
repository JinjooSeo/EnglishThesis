\ktitle{ALICE에서 수행한 양성자-양성자 충돌에서의\\매력쿼크를 포함한 중입자의 생성량과\\매력쿼크 붕괴 비율 측정} 
\etitle{Charm baryon production and fragmentation fractions \\ in pp collisions with ALICE}
\advisor{권 민 정} 
\kname{서 진 주} 
\ename{Jinjoo Seo} 
\coverdate{2023}{2}
\refereesigndate{2023}{2}{1}
\refereeA{윤 진 희}
\refereeC{권 민 정} 
\refereeD{조 성 태}
\refereeE{Rachid Guernane}
\refereeB{Andrea Dubla}


\pagenumbering{roman}

\title{이학박사학위 논문}
\maketitle

\rhead{\thepage}
\cfoot{}
\rfoot{„}

{\footnotesize \tableofcontents{} \newpage{}} {\footnotesize \par}

\begin{singlespace}
{\small \listoffigures 
\newpage{}}{\small \par}

{\small \listoftables
}{\small \par}
\end{singlespace}

% 초록 %
\newpage{}
\thispagestyle{empty}
\begin{singlespace}
\small \bf CHARM BARYON PRODUCTION
\end{singlespace}


\begin{center}
{Charm baryon production and fragmentation fractions\\in pp collisions with ALICE}\\
\vskip 0.8cm
\par\end{center}

\begin{center}
Jinjoo Seo{\Large }\\
\vskip 0.3cm
\par\end{center}

\begin{center}
Department of Physics, Inha University, Korea{\large }\\
\vskip 0.8cm
\par\end{center}

\begin{center}
{\small \bf Abstract}
\par\end{center}
{\small \par}

%\begin{center}
%\vskip 0.8cm
%\par\end{center}
{\small
\noindent The measurements of $\Lambda_\mathrm{c}^+$, $\Xi_\mathrm{c}^{0,+}$, $\Sigma_\mathrm{c}^{0,++}$, and the first measurement of $\Omega_\mathrm{c}^0$ baryons performed with the ALICE detector at midrapidity in pp collisions at $\sqrt{s}=5.02$ and $13\,\rm TeV$ were measured.
Also, the first measurements of the total charm cross section at midrapidity and the fragmentation fractions at midrapidity in pp collisions at the LHC including the charm baryons were measured.
The charm fragmentation fractions in pp collisions at \tevf were measured for the first time, showing that charm fragmentation was not universal across collision systems.
}{\small \par}

\begin{center}
{\small \bf{\it Keywords:}}
{\small Heavy-Ion Collisions, Quark-gluon plasma,\\Heavy quark, Charm quark, Charmed baryon}
\par\end{center}
{\small \par}

\newpage{}
\thispagestyle{empty}
\begin{center}
    { \bf Acknowledgments}\\
    \vskip 0.8cm
\par\end{center}

{\small
Fisrt of all, I would like to thank to Prof. Kweon, who is my supervior. 
She supports to me to keep going the analysis and takes care of me to stand in this field.
Prof. Yoon also supports to me giving useful theoritical explanations.
We, Stefano Trogolo and I, did lots of works together, related to ALICE 3 LoI. 
I would like to take this opportunity to say thank you.
}{\small \par}