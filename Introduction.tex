%!TEX root = thesis.tex
\chapter{Introduction}
\section{Scientific motivation}
\subsection{Quark Gluon Plasma}
The Standard Model of particle physics describes the fundamental constituents of matter and the laws governing their interactions.
It also accounts for how collective phenomena and equilibrium properties of matter arise from elementary interactions. 
Theory makes quantitative statements about the equation of state of Standard Model matter, about the nature of the electroweak and strong phase transitions, and about the fundamental properties such as transport coefficients and relaxation times.
In addition, there has been considerable theoretical progress in describing how out-of-equilibrium evolution drives non-abelian matter towards equilibrium.
Collisions of nuclei at ultra-relativistic energies offer a unique possibility for testing some key facets of the rich high-temperature thermodynamics of the Standard Model in laboratory-based experiments.
They test the strong interaction sector of the Standard Model at energy densities at which partonic degrees of freedom dominate equilibration processes.
They thus give access to the partonic dynamics that drives fundamental non-abelian matter towards equilibrium and that determines the properties of the QCD high temperature phase, the Quark Gluon Plasma~\cite{wong1994introduction, wikipedia_2021}.

The field of ultra-relativistic nuclear collisions has seen enormous progress since the mid-eighties, from the first signals of colour deconfinement at the SPS to the evidence, at RHIC, for a strongly-coupled QCD medium that quenches the momenta of hard partons~\cite{Reidt:2151986}.
Nuclear collisions at the LHC offer an almost ideal environment for a broad programme of characterization of the properties of this unique state of matter~\cite{Chatrchyan:2011pw}.
Besides providing access to the highest-temperature, longest-lived experimentally accessible QCD medium, they also provide an abundant supply of self-calibrating heavy-flavour probes.
In addition, the very low net baryon density eases the connection between experimental measurements and lattice QCD calculations significantly~\cite{Chatrchyan:2011vh}.

\subsection{Heavy quark hadronisation}
Measurements of heavy-flavour hadron production in high energy proton--proton (pp) collisions provide important tests of QCD.
The cross sections of heavy-flavour hadrons are usually computed using the factorisation approach as a convolution of three factors \cite{Collins:1985gm} the parton distribution functions (PDFs) of the incoming protons, ii) the hard-scattering cross section at partonic level, and iii) the fragmentation function of heavy quarks into a given heavy-flavour hadron.
The D- and B-meson cross sections in pp collisions at several centre-of-mass energies at the LHC~\cite{Acharya:2019mgn,nonpromptD,Sirunyan:2017xss} are described within uncertainties by perturbative QCD (pQCD) calculations~\cite{Kramer:2017gct,Helenius:2018uul,Cacciari:1998it,Cacciari:2012ny,Kniehl:2020szu}, which use fragmentation functions tuned on $\rm{e^+e^-}$ data, over a wide range of transverse momentum (\pt).
Measurements of $\rm \Lambda^{+}_c$-baryon production at midrapidity in pp collisions at the centre-of-mass energy $\sqrt{s}=$~5.02 and 7~TeV were reported by the ALICE and CMS collaborations in Refs.~\cite{Sirunyan:2019fnc,Acharya:2017kfy,Acharya:2020uqi}. The measured $\rm \Lambda^{+}_c/{\rm D^0}$ ratio is higher than previous measurements in $\rm{e^+e^-}$~\cite{Albrecht:1988an,Avery:1990bc,Gladilin:2014tba} and ${\rm e^-p}$~\cite{Chekanov:2005mm,Abramowicz:2013eja} collisions, suggesting that charm-hadronisation mechanisms are different in pp collisions at LHC energies. 
A similar observation was drawn from the measurement of the inclusive $\Xi^0_{\rm c}$-baryon production at midrapidity in pp collisions at $\sqrt{s} = 7$~TeV~\cite{Acharya:2017lwf}.
The charm-baryon cross sections measured in pp collisions are larger than next-to-leading order pQCD-based calculations~\cite{Kniehl:2020szu}, and larger than expectations from various event generators, namely POWHEG matched to PYTHIA~6 for the parton-shower and the hadronisation stages with Perugia tune~\cite{Frixione:2007nw}, PYTHIA~8 with Monash tune~\cite{Skands:2014pea}, and HERWIG~7~\cite{Bahr:2008pv}.
On the other hand, PYTHIA~8 tunes including string formation beyond the leading-colour approximation~\cite{Christiansen:2015yqa} qualitatively describe the measured $\rm \Sigma^{0,+,++}_c/{\rm D^0}$ and $\rm \Lambda^{+}_c/{\rm D^0}$ cross section ratios~\cite{SigmacLambdac,Acharya:2020uqi}, but underestimate the $\Xi^0_{\rm c}/{\rm D^0}$ ratio~\cite{Acharya:2017lwf}. 
A statistical hadronisation model (SHM)~\cite{He:2019tik} based on the charmed hadron states listed by the Particle Data Group (PDG)~\cite{Zyla:2020zbs} underestimates the $\Lambda^{+}_{\rm c}/{\rm D^0}$ ratio. However this ratio is qualitatively described by the SHM when the presence of a large set of yet-unobserved higher-mass charm-baryon states is assumed in the calculation as prescribed by the relativistic quark model (RQM) and from lattice QCD~\cite{Ebert:2011kk,Briceno:2012wt}.
The observed enhancement of the charm-baryon production can also be explained by model calculations considering hadronisation of charm quarks via coalescence in pp collisions~\cite{Song:2018tpv}.
The increased yield of charm baryons makes it mandatory to include their contribution for an accurate measurement of the ${\rm c\overline{c}}$ production cross section in pp collisions at the LHC~\cite{Acharya:2017jgo} and further provide evidence that the assumption of universality (colliding-system independence) of parton-to-hadron fragmentation is not valid.

